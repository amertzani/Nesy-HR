\documentclass[11pt,a4paper]{article}
\usepackage[utf8]{inputenc}
\usepackage[T1]{fontenc}
\usepackage{graphicx}
\usepackage{amsmath}
\usepackage{amssymb}
\usepackage{hyperref}
\usepackage{geometry}
\usepackage{float}
\usepackage{caption}
\usepackage{xcolor}

\geometry{margin=1in}

\title{Use Case: Comprehensive HR Decision Support\\
\large{Demonstrating Knowledge Graph, Statistics, Insights, LLM, and Agent Architecture}}
\author{}
\date{}

\begin{document}

\maketitle

\section{Use Case Overview}

This use case demonstrates all key system features through a realistic HR scenario. An HR manager analyzes department performance, statistical relationships, and employee information, showing how the knowledge graph, statistics agent, operational insights, LLM, and multi-agent architecture work together to provide transparent, traceable responses.

\subsection{Scenario}

\textbf{User}: HR Manager preparing for quarterly review\\
\textbf{Information Needs}: Department metrics, correlations, employee data, system transparency

\section{Query Demonstrations}

\subsection{Query 1: Knowledge Graph Retrieval}

\textbf{Query}: "What is John Smith's salary and department?"

\textbf{Processing}: Orchestrator → Query Processor → Knowledge Graph retrieval

\textbf{Response}: 
\begin{quote}
John Smith has a salary of \$75,000 and works in the Engineering department.

\textbf{Evidence from Knowledge Graph:}
\begin{enumerate}
    \item John Smith → has\_salary → \$75,000 [Source: employees.csv]
    \item John Smith → has\_department → Engineering [Source: employees.csv]
\end{enumerate}
\end{quote}

\textbf{Figure 1: Chat Interface - Knowledge Graph Query}
\begin{quote}
\textit{[Screenshot: Chat interface showing user query "What is John Smith's salary and department?" and system response with evidence facts. Evidence section displays two facts with source document attribution. Routing information shows "Query routed via orchestrator: Found employee in document worker agent (worker\_001)".]}
\end{quote}

\textbf{Figure 2: Knowledge Base View}
\begin{quote}
\textit{[Screenshot: Knowledge Base table filtered to "John Smith" showing facts with columns: Subject, Predicate, Object, Source Document, Agent ID. Three facts visible: salary, department, manager. Source document "employees.csv" and agent "worker\_001" shown for all facts. Filter controls visible at top.]}
\end{quote}

\textbf{Figure 3: Graph Visualization}
\begin{quote}
\textit{[Screenshot: Interactive force-directed graph with "John Smith" node highlighted in center, connected to "75000" (has\_salary), "Engineering" (has\_department), and "Jane Doe" (has\_manager). Hover tooltip shows source information. Graph controls for zoom/pan visible.]}
\end{quote}

\subsection{Query 2: Statistical Analysis}

\textbf{Query}: "What is the correlation between salary and performance rating?"

\textbf{Processing}: Orchestrator → Statistics Agent → Correlation facts from knowledge graph

\textbf{Response}:
\begin{quote}
The correlation between salary and performance rating is -0.64, indicating a moderate negative relationship.

\textbf{Evidence from Knowledge Graph:}
\begin{enumerate}
    \item correlation\_salary\_performance\_rating → has\_correlation\_value → -0.64 [Source: employees.csv]
    \item salary → has\_mean → 65234.5 [Source: employees.csv]
    \item performance\_rating → has\_mean → 3.2 [Source: employees.csv]
\end{enumerate}
\end{quote}

\textbf{Figure 4: Statistics Dashboard}
\begin{quote}
\textit{[Screenshot: Statistics page showing correlation heatmap with "salary" × "performance\_rating" cell highlighted (-0.64). Descriptive statistics table showing mean, median, std for both variables. Distribution charts visible. Source attribution: "Statistics computed from employees.csv".]}
\end{quote}

\subsection{Query 3: Operational Insights}

\textbf{Query}: "What is the average salary by department?"

\textbf{Processing}: Orchestrator → Operational Agent → Pre-computed groupby insights

\textbf{Response}:
\begin{quote}
Average salaries by department:
\begin{itemize}
    \item Engineering: \$72,000 (45 employees)
    \item Sales: \$68,000 (32 employees)
    \item Marketing: \$65,000 (28 employees)
    \item HR: \$62,000 (15 employees)
\end{itemize}

\textbf{Evidence from Knowledge Graph:}
\begin{enumerate}
    \item Engineering → has\_avg\_salary → 72000 [Source: employees.csv]
    \item Engineering → has\_employee\_count → 45 [Source: employees.csv]
    \item Sales → has\_avg\_salary → 68000 [Source: employees.csv]
    \item Sales → has\_employee\_count → 32 [Source: employees.csv]
\end{enumerate}
\end{quote}

\textbf{Figure 5: Insights Page}
\begin{quote}
\textit{[Screenshot: Insights page showing "Department Performance Metrics" table with columns: Department, Employee Count, Avg Salary, Avg Performance, Avg Absences. Rows for Engineering, Sales, Marketing, HR with respective metrics. Additional sections: "By Manager", "By Recruitment Source". Source attribution visible.]}
\end{quote}

\subsection{Query 4: LLM Conversational Query}

\textbf{Query}: "Which department has the highest employee turnover and what might be the reasons?"

\textbf{Processing}: Orchestrator → LLM Agent → Multi-fact retrieval and synthesis

\textbf{Response}:
\begin{quote}
The Sales department has the highest turnover rate at 18\%. Contributing factors:
\begin{itemize}
    \item Higher absence rate: 3.2 vs company average 2.4
    \item Lower engagement: 2.8 vs company average 3.2
    \item Performance variance indicating inconsistent management
\end{itemize}

\textbf{Evidence from Knowledge Graph:}
\begin{enumerate}
    \item Sales → has\_termination\_rate → 0.18 [Source: employees.csv]
    \item Sales → has\_avg\_absences → 3.2 [Source: employees.csv]
    \item Sales → has\_avg\_engagement → 2.8 [Source: employees.csv]
    \item Sales → has\_avg\_performance → 3.5 [Source: employees.csv]
\end{enumerate}
\end{quote}

\textbf{Figure 6: Chat Interface - LLM Query}
\begin{quote}
\textit{[Screenshot: Chat interface showing complex query and multi-paragraph LLM response with analysis. Evidence section shows 4 supporting facts. Response demonstrates synthesis of multiple facts into coherent insights. Routing information: "LLM Agent with KG context".]}
\end{quote}

\subsection{Agent Architecture Visualization}

\textbf{User Action}: Navigates to Agent Architecture page

\textbf{Figure 7: Agent Network}
\begin{quote}
\textit{[Screenshot: Agent Architecture page showing network visualization with Orchestrator Agent (central, purple), connected to LLM Agent, Statistics Agent, Operational Agent, Query Processor. All agents connected to Knowledge Graph Agent (center, yellow). Document Agents connected to Worker Agents. Arrows show data flow. Agent status cards below showing: Orchestrator (Active), LLM (Active, 47 queries), Statistics (Active, 28 correlations), Operational (Active, 12 insights), KG (Active, 1,247 facts), Document Agent (Completed, 1,247 facts), Worker Agents (4, all Completed). Color coding: Green=Active, Blue=Processing, Gray=Idle.]}
\end{quote}

\section{Key Features Demonstrated}

\begin{itemize}
    \item \textbf{Knowledge Graph}: Direct fact retrieval with complete provenance (Query 1)
    \item \textbf{Statistics}: Pre-computed correlations stored as facts (Query 2)
    \item \textbf{Operational Insights}: Groupby aggregations stored as traceable facts (Query 3)
    \item \textbf{LLM}: Natural language synthesis with evidence facts (Query 4)
    \item \textbf{Agent Architecture}: Complete system visibility and transparency (Agent view)
\end{itemize}

\section{Transparency Mechanisms}

All queries demonstrate:
\begin{enumerate}
    \item \textbf{Evidence Display}: Every response includes supporting facts
    \item \textbf{Source Attribution}: All facts linked to source documents
    \item \textbf{Agent Visibility}: Routing information shows processing agents
    \item \textbf{Knowledge Graph Access}: Users can verify facts directly
    \item \textbf{Complete Provenance}: Full chain from source to response
\end{enumerate}

\section{Conclusion}

This use case demonstrates that the system provides transparent, traceable responses across all query types, enabling HR professionals to make evidence-based decisions with complete confidence in system reasoning.

\appendix
% Use Case Appendix - Additional Details
% Include this after technical_sections_appendix.tex

\section{Use Case Appendix: Additional Details}

\subsection{Complete Provenance Chains}

\subsubsection{Knowledge Graph Query Provenance}
\begin{enumerate}
    \item Source Document: employees.csv (uploaded 2024-01-15T10:30:00)
    \item Processing: Document Agent (doc\_001) → Worker Agent (worker\_001)
    \item Extraction: CSV row processing, confidence 1.0
    \item Storage: RDF triple (John\_Smith, has\_salary, 75000) with metadata
    \item Query: Orchestrator → Query Processor → Knowledge Graph retrieval
    \item Response: LLM Agent formats with evidence facts
    \item Display: Chat interface with clickable source links
\end{enumerate}

\subsubsection{Statistical Query Provenance}
\begin{enumerate}
    \item Source Document: employees.csv
    \item Processing: Statistics Agent computes Pearson correlation during document processing
    \item Storage: Correlation fact stored in knowledge graph
    \item Query: Orchestrator → Statistics Agent → Knowledge Graph retrieval
    \item Response: LLM Agent explains statistical findings
    \item Display: Chat response + Statistics Dashboard visualization
\end{enumerate}

\subsubsection{Operational Query Provenance}
\begin{enumerate}
    \item Source Document: employees.csv
    \item Processing: Operational Agent computes groupby aggregations
    \item Storage: Operational facts stored in knowledge graph
    \item Query: Orchestrator → Operational Agent → Knowledge Graph retrieval
    \item Response: LLM Agent formats insights with evidence
    \item Display: Chat response + Insights Page table
\end{enumerate}

\subsubsection{LLM Query Provenance}
\begin{enumerate}
    \item Source: Multiple facts from knowledge graph
    \item Processing: LLM Agent retrieves relevant facts using relevance ranking
    \item Synthesis: LLM combines facts into coherent analysis
    \item Response: Natural language with evidence facts
    \item Display: Chat interface with comprehensive answer
\end{enumerate}

\subsection{Additional Screenshot Descriptions}

\subsubsection{Source Document View}
\textit{[Screenshot: Documents page showing "employees.csv" with metadata: Upload timestamp, Total facts extracted (1,247), Processing status (Completed), Processing agent (doc\_001). Link to "View All Facts" visible. Sample facts list showing first 10 facts from document.]}

\subsubsection{Statistics Correlation Matrix}
\textit{[Screenshot: Full correlation matrix heatmap showing all pairwise correlations between numeric columns. Color scale: Red (negative), White (zero), Blue (positive). "salary" × "performance\_rating" cell highlighted. Interactive tooltips showing exact correlation values on hover.]}

\subsubsection{Insights Manager View}
\textit{[Screenshot: Insights page "By Manager" section showing table with columns: Manager Name, Employee Count, Avg Salary, Avg Performance, Avg Absences. Multiple manager rows visible. Source attribution and refresh button visible.]}

\subsubsection{Graph Node Details}
\textit{[Screenshot: Graph visualization with "John Smith" node selected, showing detailed sidebar panel with: Node properties (type: Employee), All connections (3 edges), Source information (employees.csv, worker\_001), Related entities list, Edit/Delete buttons.]}

\subsection{Query Type Coverage}

The system handles four main query types:
\begin{itemize}
    \item \textbf{Structured Queries}: Direct fact retrieval from knowledge graph
    \item \textbf{Statistical Queries}: Correlation and distribution analysis
    \item \textbf{Operational Queries}: Groupby and aggregation insights
    \item \textbf{Conversational Queries}: Natural language with multi-fact synthesis
\end{itemize}

Each query type maintains transparency through evidence facts and source attribution.

\subsection{Agent Communication Patterns}

\begin{itemize}
    \item \textbf{Centralized Knowledge Graph}: All agents write to and read from the same knowledge graph
    \item \textbf{Orchestrator Routing}: Single entry point routes queries to appropriate agents
    \item \textbf{Parallel Processing}: Document agents create worker agents that process chunks concurrently
    \item \textbf{Specialized Processing}: Each agent type handles specific tasks
    \item \textbf{Transparent Status}: All agents report status visible to users
\end{itemize}

\subsection{User Interface Features}

\begin{itemize}
    \item \textbf{Chat Interface}: Natural language interaction with evidence display
    \item \textbf{Knowledge Base}: Table view for fact inspection and filtering
    \item \textbf{Graph Visualization}: Interactive exploration of entity relationships
    \item \textbf{Statistics Dashboard}: Correlation matrices and descriptive statistics
    \item \textbf{Insights Page}: Operational metrics and aggregations
    \item \textbf{Agent Architecture}: System transparency and status monitoring
    \item \textbf{Documents Page}: Source document management and verification
\end{itemize}



\end{document}
